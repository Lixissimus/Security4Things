% v2-acmtog-sample.tex, dated March 7 2012
% This is a sample file for ACM Transactions on Graphics
%
% Compilation using 'acmtog.cls' - version 1.2 (March 2012), Aptara Inc.
% (c) 2010 Association for Computing Machinery (ACM)
%
% Questions/Suggestions/Feedback should be addressed to => "acmtexsupport@aptaracorp.com".
% Users can also go through the FAQs available on the journal's submission webpage.
%
% Steps to compile: latex, bibtex, latex latex
%
% For tracking purposes => this is v1.2 - March 2012
\documentclass{sig-alternate} % v 2.5

% --- Author Metadata here ---
\conferenceinfo{Sample Conferenve}{'15 Potsdam, Germany}
%\CopyrightYear{2007} % Allows default copyright year (20XX) to be over-ridden - IF NEED BE.
%\crdata{0-12345-67-8/90/01}  % Allows default copyright data (0-89791-88-6/97/05) to be over-ridden - IF NEED BE.
% --- End of Author Metadata ---

\begin{document}

\title{Project Light}

\numberofauthors{5}

\author{
\alignauthor
Cornelius Bock
%
\alignauthor
Daniel Werner
%
\alignauthor
Felix Wolff
%
\and
\alignauthor
Christoph Meinel
%
\alignauthor
Konrad-Felix Krentz
}

\maketitle

\begin{abstract}
Lorem ipsum dolor sit amet, consectetur adipisicing elit, sed do eiusmod
tempor incididunt ut labore et dolore magna aliqua. Ut enim ad minim veniam,
quis nostrud exercitation ullamco laboris nisi ut aliquip ex ea commodo
consequat. Duis aute irure dolor in reprehenderit in voluptate velit esse
cillum dolore eu fugiat nulla pariatur. Excepteur sint occaecat cupidatat non
proident, sunt in culpa qui officia deserunt mollit anim id est laborum.
\end{abstract}

\category{K.6.5}{Management Of Computing And Information Systems}{Security}

\terms{Internet of Things, Security}

\keywords{awesome, keywords, go, here}

\section{Introduction}
\label{sec:introduction}

\begin{itemize}
	\item short introduction internet of things
	\item some words about contiki
	\item current security and key distribution situation (akes)
	\item idea of transmitting keying material using light and phone app, why light, alternative solutions
	\item setup (android device/app, mote with light sensor)
\end{itemize}


\section{Related Work}
\label{sec:related_work}

\begin{itemize}
	\item LiDSN: A Method to Deploy Wireless Sensor Networks Securely based on light communication
	\item ...
\end{itemize}


\section{Communication Protocol}
\label{sec:communication_protocol}

\begin{itemize}
	\item phases of protocol
	\begin{itemize}
		\item calibration
		\item synchronization
		\item initialization
		\item data sending
		\item validation
	\end{itemize}
\end{itemize}



\section{Implementation}
\label{sec:contiki_integration}

\begin{itemize}
	\item hamming code, crc32 checksum
	\item Android app - functionality, flashlight, limitations
	\item contiki process - functionality, rtimer?, kmeans?, limitations
	\item driver wrapper to delay link layer initialization until key is initialized
\end{itemize}



\section{Results and Discussion}
\label{sec:results_and_discussion}

\begin{itemize}
	\item measure and evaluate maximum transmission speed, error rate
	\item is hamming code necessary (worth the doubled length) or CRC alone sufficient
	\item does the transmission rate have to be variable?
	\item "keep the phone busy" (wake up more often to prevent cpu from sleeping)
\end{itemize}



\section{Future Work}
\label{sec:future_work}

\begin{itemize}
	\item bidirectional communication using the LEDs of the mote (and maybe camera)
	\item 802.15.4 dongle to verify transmission (and do further initialization)
	\item initialize several mote together, separate synchronization and data sending
\end{itemize}

% Bibliography
\bibliographystyle{ACM-Reference-Format-Journals}
\bibliography{paper}

\end{document}
